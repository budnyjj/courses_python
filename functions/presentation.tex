\documentclass[hyperref={pdftex,unicode}]{beamer}

\usepackage[T2A]{fontenc}
\usepackage[utf8]{inputenc}
\usepackage[russian]{babel}
\usepackage{cmap}

\usepackage{xcolor}

% \usepackage{helvet}
\usepackage{pscyr}

\usepackage{multicol}

% \usepackage{amssymb,amsfonts,amsmath,mathtext}
% \usepackage{cite,enumerate,float}

\usepackage{listings}

\graphicspath{{images/}}
\usetheme{boxes}
\usecolortheme{whale}
\useinnertheme{default}
\useoutertheme{default}
\usefonttheme{professionalfonts}

\definecolor{Blue}{RGB}{55,110,160}
\definecolor{Yellow}{RGB}{255,210,65}
\definecolor{Green}{RGB}{75,120,60}
\definecolor{LightGray}{RGB}{205,205,205}

\setbeamercolor*{palette primary}{fg=white,bg=Blue}

\setbeamercolor*{enumerate item}{fg=Yellow}
\setbeamercolor*{enumerate subitem}{fg=Yellow}
\setbeamercolor*{enumerate subsubitem}{fg=Yellow}

\setbeamercolor*{itemize item}{fg=Yellow}
\setbeamercolor*{itemize subitem}{fg=Yellow}
\setbeamercolor*{itemize subsubitem}{fg=Yellow}

\lstset{
  language=python,
  basicstyle=\small\ttfamily,
  keywordstyle=\bfseries\color{Blue},
  commentstyle=\color{Green},
  numberstyle=\footnotesize\color{LightGray},
  numbersep=0.8em,
  stepnumber=1,
  keepspaces=true,
  breaklines=true,
  aboveskip=0.5\baselineskip,
  belowskip=0.5\baselineskip}


\title{Python: functions}
\author{meequz@gmail.com \\ budnyjj@pirates.by}
\date{}

\begin{document}

\begin{frame}
  \maketitle
\end{frame}

\begin{frame}{Сегодня}
  \begin{itemize}
    \item Функции
    \item Модули
    \item Стандартная библиотека
  \end{itemize}
\end{frame}

\begin{frame}[fragile]{Определение и вызов функции}
    \begin{lstlisting}[numbers=left]
def cube(x):
    return x ** 3

print(cube(12))

print(cube)
print(isinstance(cube, object))
print(dir(cube))
    \end{lstlisting}
\end{frame}

\begin{frame}[fragile]{Передача аргументов в функцию}
      \begin{lstlisting}[numbers=left]
def modify(string, lst):
    string = "new " + string
    for idx, val in enumerate(lst):
        lst[idx] = "new " + val

names = ["cat", "book", "cinema"]
s = "machine"

print "BEFORE MODIFY()"
print names
print s

modify(s, names)

print "AFTER MODIFY()"
print names
print s
    \end{lstlisting}
\end{frame}

\begin{frame}{Типы аргументов}
  \begin{itemize}
  \item Обязательные
  \item Именованные
  \item По умолчанию
  \item Переменной длины
  \item Ключевые
  \end{itemize}
\end{frame}

\begin{frame}[fragile]{Значение аргументов по умолчанию\footnote[frame]{
      http://zetcode.com/lang/python/functions/
}}
  \begin{lstlisting}[numbers=left]
def power(x, y=2):
    r = 1
    for i in range(y):
       r = r * x
    return r

print power(3)
print power(3, 3)
print power(5, 5)
  \end{lstlisting}
\end{frame}

\begin{frame}[fragile]{Задание значения аргумента по имени}
  \begin{lstlisting}[numbers=left]
def display(name, age, sex="M"):
   print "Name: ", name
   print "Age: ", age
   print "Sex: ", sex

display("Lary", 43, "M")
display("Lary", age=43)
display("Lary", age=43, sex="M")
display(age=43, name="Lary", sex="M")

display(age=24, name="Joan", "F")         # error
\end{lstlisting}
\end{frame}

\begin{frame}[fragile]{Список аргументов переменной длины}
  \begin{lstlisting}[numbers=left]
print sum

def sum(*args):
   '''Function returns the sum of all values'''
   r = 0
   for i in args:
      r += i
   return r

print sum
print sum.__doc__
print sum(1, 2, 3)
print sum(1, 2, 3, 4, 5)
  \end{lstlisting}
\end{frame}

\begin{frame}[fragile]{Ключевые аргументы}
  \begin{lstlisting}[numbers=left]
def display(**details):
   for i in details:
      print "%s: %s" % (i, details[i])

display(name="Lary", age=43, sex="M")
  \end{lstlisting}
\end{frame}

\begin{frame}[fragile]{Особенности передачи аргументов}
  \begin{lstlisting}[numbers=left]
def func(a,b,c,d=False,*args,**kwargs):
    print a, b, c, d, args, kwargs

func(*[1,2,3,4,5], **{'6':7})
func(*[1,2,3,], **{'d':7})
func(1, 2, *[3,], **{'d':7})
  \end{lstlisting}
\end{frame}

\begin{frame}[fragile]{Область видимости}
  \begin{lstlisting}[numbers=left]
name = "Jack"

def f():
   # global name 
   name = "Robert"
   print "Within function:", name
   print locals()
   print globals()

print "Outside function:", name
f()
print "Outside function:", name
\end{lstlisting}
\end{frame}

\begin{frame}{Объекты первого класса}
  ``Объектами первого класса в контексте конкретного языка программирования называются сущности,
  которые могут быть переданы как параметр, возвращены из функции, присвоены переменной''\footnote[frame]{
    http://en.wikipedia.org/wiki/First-class\_citizen}.
\end{frame}

\begin{frame}[fragile]{Функция как объект первого класса}
  \begin{lstlisting}[numbers=left]
def square(x):
    return x ** 2

s = square
print s(5)

def ff(f, x):
    return f(f(x) - 1)

print ff(s, 5)
\end{lstlisting}
\end{frame}

\begin{frame}[fragile]{Анонимные функции}
  \begin{lstlisting}[numbers=left]
lowercase = lambda x: x.lower()
print_assign = lambda name, value:\
                      name + '=' + str(value)
adder = lambda x, y: x+y

print lowercase("THETA")
print print_assign("two", 2)
print adder(2, 3)
\end{lstlisting}
\end{frame}

\begin{frame}[fragile]{Simple is better than complex}
  \begin{lstlisting}
items = [("one", 1), ("two", 2), ("three", 3)]
# 1 + 2 + 3 = ?
\end{lstlisting}

\begin{itemize}
\item
  \begin{lstlisting}[numbers=right]
total = reduce(lambda a, b:\
              (0, a[1] + b[1]),items)[1]
\end{lstlisting}

\item
\begin{lstlisting}[numbers=right]
def combine (a, b):
    return 0, a[1] + b[1]
total = reduce(combine, items)[1]
\end{lstlisting}

\item
\begin{lstlisting}[numbers=right]
total = 0
for a, b in items:
    total += b
\end{lstlisting}

\item
\begin{lstlisting}[numbers=right]
total = sum(b for a,b in items)
\end{lstlisting}
\end{itemize}
\end{frame}

\begin{frame}[fragile]{Декораторы}
\begin{minipage}{0.4\textwidth}
  \begin{lstlisting}[]
def func(...)
...
def wrapper(...)
...
f = wrapper(func)
  \end{lstlisting}
\end{minipage}
\Large{vs}
\hfill
\begin{minipage}{0.45\textwidth}
  \begin{lstlisting}
def wrapper(...)
...
@wrapper
def func(...)
...
  \end{lstlisting}   
\end{minipage}
\end{frame}

\begin{frame}[fragile]{Пример декоратора}
  \begin{lstlisting}[numbers=left]
import time

def timer(f):
    def tmp(*args, **kwargs):
        t = time.time()
        res = f(*args, **kwargs)
        print "Time of function evaluation: \
               {:f}".format(time.time()-t)
        return res
    return tmp

@timer
def func(x, y):
    return x + y

print func(1, 2)
\end{lstlisting}
\end{frame}


\begin{frame}[fragile]{Модуль fibo.py}
  \begin{lstlisting}[basicstyle=\footnotesize\ttfamily,numbers=left]
# Fibonacci numbers module

def fib(n):    # write Fibonacci series up to n
    a, b = 0, 1
    while b < n:
        print b,
        a, b = b, a+b

def fib2(n): # return Fibonacci series up to n
    result = []
    a, b = 0, 1
    while b < n:
        result.append(b)
        a, b = b, a+b
    return result

if __name__ == "__main__":
    import sys
    fib(int(sys.argv[1]))
\end{lstlisting}
\end{frame}

\begin{frame}{Импорт модуля}
  \begin{itemize}
    \item import fibo
    \item import fibo as f
    \item from fibo import fib
    \item from fibo import *
  \end{itemize}
\end{frame}

\begin{frame}[fragile]{Использование fibo.py}
\begin{itemize}
\item В качестве модуля:
  \begin{lstlisting}[numbers=left]
import fibo
dir(fibo)
fib = fibo.fib(500)
  \end{lstlisting}
\item В качестве скрипта:
  \begin{lstlisting}[language=bash,numbers=left]
python2 fibo.py 50
  \end{lstlisting}
\end{itemize}
\end{frame}

\begin{frame}[fragile]{import sound.effects.echo}
  \begin{lstlisting}[basicstyle=\scriptsize\ttfamily]
sound/                # Top-level package
      __init__.py     # Initialize the sound package
      formats/        # Subpackage for file format conversions
              __init__.py 
              wavread.py  
              wavwrite.py 
              aiffread.py 
              aiffwrite.py
              ...         
      effects/        # Subpackage for sound effects
              __init__.py 
              echo.py     
              surround.py 
              reverse.py  
              ...         
      filters/        # Subpackage for filters
              __init__.py
              equalizer.py
              vocoder.py
              karaoke.py
              ...
  \end{lstlisting}
\end{frame}

\begin{frame}{Обзор содержимого стандартной библиотеки}
\footnotesize{  
  \begin{minipage}{0.5\linewidth}
    \centering Организация ВП
    \begin{multicols}{2}
    \begin{itemize}
    \item \textbf{re}
    \item difflib
    \item datetime
    \item calendar
    \item \textbf{collections}
    \item heapq
    \item bisect
    \item array
    \item Queue
    \item mutex
    \item math, cmath
    \item random
    \item fraction
    \item pprint
    \item \textbf{itertools}
    \item functools
    \item hashlib
    \item threading
    \end{itemize}    
  \end{multicols}
  \end{minipage}
  \hfill
  \begin{minipage}{0.25\linewidth}
    \centering Ввод и вывод
    \begin{itemize}
    \item pickle
    \item gdbm
    \item zlib
    \item csv
    \item email
    \item json
    \item HTMLParser
    \item xml.*
    \item webbrowser
    \item urllib[2]
    \item httplib
    \end{itemize}
  \end{minipage}
  \hfill
  \begin{minipage}{0.2\linewidth}
    \centering Поддержка, \\
      взаимодействие \\
      с платформой
    \begin{itemize}
    \item os
    \item os.path
    \item argparse
    \item gettext
    \item locale
    \item pydoc
    \item logging
    \item unittest
    \item pdb    
    \end{itemize}
  \end{minipage}
}
\end{frame}

\begin{frame}{Полезные ссылки}
  \begin{itemize}
    \item http://zetcode.com/lang/python/functions/
    \item https://docs.python.org/2/library/functions.html
    \item https://docs.python.org/2/howto/functional.html
    \item https://docs.python.org/2/tutorial/modules.html
    \item https://docs.python.org/2.7/library/
  \end{itemize}
\end{frame}

\begin{frame}{Продолжение следует\dots}
  Через неделю --- классы и исключения:
  \begin{itemize}
  \item Аттрибуты, методы объектов
  \item Инкапсуляция
  \item Наследование
  \item Генерация и обработка исключений
  \end{itemize}
\end{frame}


\end{document}