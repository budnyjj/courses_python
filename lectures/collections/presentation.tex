\documentclass[hyperref={pdftex,unicode}]{beamer}

\usepackage[T2A]{fontenc}
\usepackage[utf8]{inputenc}
\usepackage[russian]{babel}
\usepackage{cmap}

\usepackage{xcolor}

% \usepackage{helvet}
\usepackage{pscyr}

\usepackage{multicol}

% \usepackage{amssymb,amsfonts,amsmath,mathtext}
% \usepackage{cite,enumerate,float}

\usepackage{listings}

\graphicspath{{images/}}
\usetheme{boxes}
\usecolortheme{whale}
\useinnertheme{default}
\useoutertheme{default}
\usefonttheme{professionalfonts}

\definecolor{Blue}{RGB}{55,110,160}
\definecolor{Yellow}{RGB}{255,210,65}
\definecolor{Green}{RGB}{75,120,60}
\definecolor{LightGray}{RGB}{205,205,205}

\setbeamercolor*{palette primary}{fg=white,bg=Blue}

\setbeamercolor*{enumerate item}{fg=Yellow}
\setbeamercolor*{enumerate subitem}{fg=Yellow}
\setbeamercolor*{enumerate subsubitem}{fg=Yellow}

\setbeamercolor*{itemize item}{fg=Yellow}
\setbeamercolor*{itemize subitem}{fg=Yellow}
\setbeamercolor*{itemize subsubitem}{fg=Yellow}

\lstset{
  language=python,
  basicstyle=\small\ttfamily,
  keywordstyle=\bfseries\color{Blue},
  commentstyle=\color{Green},
  numberstyle=\footnotesize\color{LightGray},
  numbersep=0.8em,
  stepnumber=1,
  keepspaces=true,
  breaklines=true,
  aboveskip=0.5\baselineskip,
  belowskip=0.5\baselineskip}


\title{Python: functions}
\author{meequz@gmail.com \\ budnyjj@pirates.by}
\date{}

\begin{document}

\begin{frame}
  \maketitle
\end{frame}

\begin{frame}{Сегодня}
  \begin{itemize}
  \item Строки
  \item Списки
  \item Словари
  \item Кортежи
  \item Множества
  \end{itemize}
\end{frame}

\begin{frame}{Строки и кодировки}
  \begin{itemize}
    \item в байте восемь битов
    \item ASCII --- 128 символов
    \item Unicode 6.0 --- 109384 символов (!)
  \end{itemize}
\end{frame}

\begin{frame}{Строки в Python 2}
  \begin{minipage}{0.55\linewidth}
    \textbf{str} ('string'):
    \begin{itemize}
    \item представляет собой последовательность байтов
    \item требует меньше памяти для хранения
    \end{itemize}
  \end{minipage}
  \hfill
  \begin{minipage}{0.4\linewidth}
    \textbf{unicode} (u'строка'):
    \begin{itemize}
    \item представляет собой текст
    \item подходит для локализации
    \end{itemize}
  \end{minipage}
\end{frame}

\begin{frame}{Строки в Python 3}
  \begin{center}
    \textbf{str} ('some data') $ \Rightarrow $ \textbf{bytes} (b'some data') \\
    \vspace{1em}
    \textbf{unicode} (u'строка') $ \Rightarrow $ \textbf{str} ('строка')
  \end{center}
\end{frame}

\begin{frame}[fragile]{Объявление строк}
\begin{lstlisting}[basicstyle=\footnotesize\ttfamily,numbers=left]
ordinary_string = 'hello'     # == "hello"
print(ordinary_string)

unicode_string  = u'hello'
print(unicode_string)

multiline_string = '''
  This is multiline text.
  It's stored in variable as written.
'''
print(multiline_string)
\end{lstlisting}
\end{frame}

\begin{frame}[fragile]{String are immutables}
\begin{lstlisting}[basicstyle=\footnotesize\ttfamily,numbers=left]
  a = 'string'
  ref_a = a
  print(a)
  print(ref_a)

  a = 'new ' + a
  print(a)
  print(ref_a)
\end{lstlisting}
\end{frame}

\begin{frame}[fragile]{String: методы}
\begin{lstlisting}[basicstyle=\footnotesize\ttfamily,numbers=left]
a = ' ...hello, my dear FrIeNdS!... '
print(a)

print(a.center(80, '_'))
print(a.count('e'))
print(a.endswith('!'))
print(a.find('my'))

print(a.isspace())

print(a.lower())
print(a.upper())
print(a.title())

print(a.strip(' .'))
print(a.replace(',', '!'))
print(a.split(' '))
\end{lstlisting}
\end{frame}

\begin{frame}[fragile]{String: пример}
\begin{lstlisting}[basicstyle=\footnotesize\ttfamily,numbers=left]
usr_input = input("Input some words in english: ")

words_upper = []
for word in usr_input.split(' '):
    if word.isupper():
        words_upper.append(word)

print('Number of uppercase words:', len(words_upper))
print('List of them:', words_upper)
\end{lstlisting}
\end{frame}

\begin{frame}{Вопросы}
  \begin{itemize}
  \item Как создать строку, содержащую кавычки?
  \item Как создать копию строки?
  \item Как вывести строку справа-налево?
  \end{itemize}
\end{frame}

\begin{frame}[fragile]{List --- универсальный контейнер}
  \begin{lstlisting}[basicstyle=\footnotesize\ttfamily,numbers=left]
# basic
print([1, 2, 3])

# from map
def cube(x): return x*x*x
print(map(cube, range(1, 11)))

# from filter
def f(x): return x % 3 == 0 or x % 5 == 0
print(filter(f, range(2, 25)))

# list comprehension
print([x**2 for x in range(10)])
\end{lstlisting}
\end{frame}

\begin{frame}[fragile]{List: методы}
  \begin{lstlisting}[basicstyle=\footnotesize\ttfamily,numbers=left]
lst = ['a', 1, ]
print(lst)

lst.append('a')
print(lst)

lst.extend([1, 2, 3])
print(lst)

lst.insert(3, 'I')
print(lst)

lst.pop()
print(lst)

print(lst.index(1))

lst.reverse()
print(lst)
  \end{lstlisting}
\end{frame}

\begin{frame}{Вопросы}
  \begin{itemize}
  \item Как удалить элемент из списка?
  \item Как отсортировать список по своим правилам?
  \item Как создать копию списка?
  \end{itemize}
\end{frame}

\begin{frame}[fragile]{Dict --- хранилище типа ключ-значение}
  \begin{lstlisting}[basicstyle=\footnotesize\ttfamily,numbers=left]
tel = {'jack': 4098, 'sape': 4139}
tel['guido'] = 4127

print(tel)
print(tel.keys())
print(tel.values())

# dict comprehension
print({x: x**2 for x in (2, 4, 6)})
  \end{lstlisting}
\end{frame}

\begin{frame}[fragile]{Dict: обход}
  \begin{lstlisting}[basicstyle=\footnotesize\ttfamily,numbers=none]
knights = {
    'gallahad': 'the pure',
    'robin':    'the brave',
}
  \end{lstlisting}
  \begin{minipage}{0.55\linewidth}
    \begin{lstlisting}[basicstyle=\footnotesize\ttfamily,numbers=left]
for k, v in knights.items():
    print(k, v)
    \end{lstlisting}
  \end{minipage}
  \hfill
  \begin{minipage}{0.43\linewidth}
    \begin{lstlisting}[basicstyle=\footnotesize\ttfamily,numbers=left]
for k in knights():
    print(k, knights[k])
    \end{lstlisting}
  \end{minipage}
\end{frame}

\begin{frame}[fragile]{Dict: пример}
\begin{lstlisting}[basicstyle=\footnotesize\ttfamily,numbers=left]
infile = open('input.txt', 'r')
words = infile.read().split()
infile.close()

counts = {}
for w in words:
    counts[w] = counts.get(w, 0) + 1

outfile = open('output.txt', 'w')
for w in sorted(counts, key=counts.get, reverse=True):
    outfile.write("| {:20} | {} time(s) |\n".format(
    w, counts[w]))
outfile.close()
\end{lstlisting}
\end{frame}

\begin{frame}[fragile]{Нет предела совершенству}
\begin{lstlisting}[basicstyle=\footnotesize\ttfamily,numbers=left]
from collections import Counter

with open('input.txt', 'r') as infile:
    words = infile.read().split()

counts = Counter(words)

with open('output.txt', 'w') as outfile:
    for w,n in counts.most_common():
        outfile.write("| {:20} | {} time(s) |\n".format(
                      w, n))
\end{lstlisting}
\end{frame}

\begin{frame}[fragile]{Tuple --- неизменяемый список}
  \begin{lstlisting}[basicstyle=\footnotesize\ttfamily,numbers=left]
a = (1, 2)
b = tuple([3, 4])
print(a)
print(b)

a, b = b, a
print(a)
print(b)
  \end{lstlisting}
\end{frame}

\begin{frame}{Вопросы}
  \begin{itemize}
  \item Зачем нужен tuple?
  \end{itemize}
\end{frame}

\begin{frame}{Set --- математичекое множество}
  \begin{itemize}
  \item Хранит уникальные элементы
  \item Порядок не важен
  \item Определены операции над множествами
  \end{itemize}
\end{frame}

\begin{frame}[fragile]{Set: операции}
\begin{lstlisting}[basicstyle=\footnotesize\ttfamily,numbers=left]
a = set('abracadabra')
b = set('alacazam')

print a
print b

print a - b           # letters in a but not in b
print a | b           # letters in either a or b

print a & b           # letters in both a and b
print a ^ b           # letters in a or b but not both
\end{lstlisting}
\end{frame}

\begin{frame}{Вопросы}
  \begin{itemize}
  \item Как убрать дубликаты из списка?
  \end{itemize}
\end{frame}

\begin{frame}{Спасибо за внимание!}
  \begin{itemize}
  \item https://github.com/budnyjj/courses\_python
  \item https://vk.com/budnyjj
  \item budnyjj@gmail.com
  \end{itemize}
\end{frame}

\end{document}