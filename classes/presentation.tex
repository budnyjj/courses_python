\documentclass[hyperref={pdftex,unicode}]{beamer}

\input{packages}
\usetheme{boxes}
\usecolortheme{whale}
\useinnertheme{default}
\useoutertheme{default}
\usefonttheme{professionalfonts}

\definecolor{Blue}{RGB}{55,110,160}
\definecolor{Yellow}{RGB}{255,210,65}
\definecolor{Green}{RGB}{75,120,60}
\definecolor{LightGray}{RGB}{205,205,205}

\setbeamercolor*{palette primary}{fg=white,bg=Blue}

\setbeamercolor*{enumerate item}{fg=Yellow}
\setbeamercolor*{enumerate subitem}{fg=Yellow}
\setbeamercolor*{enumerate subsubitem}{fg=Yellow}

\setbeamercolor*{itemize item}{fg=Yellow}
\setbeamercolor*{itemize subitem}{fg=Yellow}
\setbeamercolor*{itemize subsubitem}{fg=Yellow}

\lstset{
  language=python,
  basicstyle=\small\ttfamily,
  keywordstyle=\bfseries\color{Blue},
  commentstyle=\color{Green},
  numberstyle=\footnotesize\color{LightGray},
  numbersep=0.8em,
  stepnumber=1,
  keepspaces=true,
  breaklines=true,
  aboveskip=0.5\baselineskip,
  belowskip=0.5\baselineskip}


\title{Python: functions}
\author{budnyjj@pirates.by}
\date{}

\begin{document}

\begin{frame}
  \maketitle
\end{frame}

\begin{frame}{Сегодня}
  \begin{itemize}
    \item ООП
      \begin{itemize}
      \item Инкапсуляция
      \item Наследование
      \item Полиморфизм
      \end{itemize}
    \item Работа с исключениями
  \end{itemize}
\end{frame}

\begin{frame}{Объектно-ориентированное программирование}
  Объект --- сущность, состоящая из \\
  \begin{itemize}
  \item данных (полей, атрибутов) и 
  \item методов обработки этих данных (методов объекта).
  \end{itemize}
\end{frame}

\begin{frame}{Пример объекта}
  \begin{minipage}{0.6\linewidth}
  Кроме прочих своих достоинств, кот
  \begin{itemize}
    \item демонстрирует характерное поведение,
    \item реагирует на сообщения,
    \item наделён унаследованными реакциями,
    \item управляет своим, вполне независимым, внутренним состоянием.\footnote[frame]{
   Roger King, My cat is object-oriented}
 \end{itemize}
\end{minipage}
\hfill
\begin{minipage}{0.3\linewidth}
 \begin{figure}[h!]
   \includegraphics[width=1\linewidth]{cat.jpg}
 \end{figure}
\end{minipage}
\end{frame}

\begin{frame}{Область применения ООП}
  Объектно-ориентированный подход хорош там, где проект подразумевает долгосрочное развитие,
  состоит из большого количества библиотек и внутренних связей.
\end{frame}


\begin{frame}[fragile]{Объявление класса [dataClass.py]}
    \begin{lstlisting}[numbers=left]
class exampleDataClass:                                                                         
    """A data example class"""
                                                         
    cls_var = "cls_data"                                                                        

    def __init__(self):                                                                         
        self.obj_var = "obj_data"   
                                                            
    def f(self):                                                                                
        return self.obj_var
    \end{lstlisting}
\end{frame}

\begin{frame}[fragile]{Атрибуты объекта [dataExample1.py]}
\begin{minipage}{0.6\linewidth}
    \begin{lstlisting}[numbers=left,basicstyle=\scriptsize\ttfamily]
import dataClass as dc                                                                          
                                                                                                
o1 = dc.exampleDataClass();                                                                     
o2 = dc.exampleDataClass();                                                                     
                                                                                                
print "obj_var is object data member:"                                                             
print "BEFORE:"                                                                                 
print "o1.obj_var: ", o1.obj_var                                                                
print "o2.obj_var:", o2.obj_var                                                                 
                                                                                                
o1.obj_var = "new_object_data"                                                                  
                                                                                                
print "AFTER:"                                                                                  
print "o1.obj_var:", o1.obj_var                                                                
print "o2.obj_var:", o2.obj_var  
    \end{lstlisting}
\end{minipage}
\hfill
\begin{minipage}{0.35\linewidth}
    \begin{lstlisting}[numbers=left,basicstyle=\tiny\ttfamily,numbers=none]
class exampleDataClass:                                                                                                      
    cls_var = "cls_data"                                                                        

    def __init__(self):                                                                         
        self.obj_var =
            "obj_data"   
                                                            
    def f(self):                                                                                
        return self.obj_var
    \end{lstlisting}
\end{minipage}
\end{frame}

\begin{frame}[fragile]{Атрибуты класса [dataExample2.py]}
\begin{minipage}{0.6\linewidth}
    \begin{lstlisting}[numbers=left,basicstyle=\scriptsize\ttfamily]
import dataClass as dc                                                                          
                                                                                                
o1 = dc.exampleDataClass();                                                                     
o2 = dc.exampleDataClass();                                                                     
                                                                                                
print "Data is object data member:"                                                             
print "BEFORE:"                                                                                 
print "o1.cls_var:", o1.cls_var                                                                 
print "o2.cls_var:", o2.cls_var                                                                 
                                                                                                
dc.exampleDataClass.cls_var = \
               "new_object_data"                                                 
                                                                                                
print "AFTER:"                                                                                  
print "o1.cls_var: ", o1.cls_var                                                                
print "o2.cls_var:", o2.cls_var 
    \end{lstlisting}
\end{minipage}
\hfill
\begin{minipage}{0.35\linewidth}
    \begin{lstlisting}[numbers=left,basicstyle=\tiny\ttfamily,numbers=none]
class exampleDataClass:                                                                                                      
    cls_var = "cls_data"                                                                        

    def __init__(self):                                                                         
        self.obj_var =
            "obj_data"   
                                                            
    def f(self):                                                                                
        return self.obj_var
    \end{lstlisting}
\end{minipage}
\end{frame}

\begin{frame}[fragile]{Изменение атрибутов объекта}
\begin{center}
  \begin{tabular}[h!]{p{0.45\linewidth} p{0.54\linewidth}}
    getattr(object, name) & var = object.name \\
    setattr(object, name, value) & object.name = value \\
    delattr(object, name) & del object.name \\
    hasattr(object, name) & \scriptsize{
      implemented by calling getattr(object, name)
      
      and seeing whether it raises an exception or not} \\
  \end{tabular}
\end{center}
\end{frame}

\begin{frame}[fragile]{Изменение атрибутов объекта [dataExample3.py]}
\begin{minipage}{0.6\linewidth}
    \begin{lstlisting}[numbers=left,basicstyle=\scriptsize\ttfamily]
import dataClass as dc                                                                          
                                                                                                
o = dc.exampleDataClass();                                                                      
print dir(o);                                                                                   
                                                                                                
del o.obj_var                                                                                   
o.new_data = "some_data"                                                                        
o.f2 = lambda x: x * 3                                                                          
                                                                                                
print dir(o);                                                                                   
print o.new_data                                                                                
print o.f2("M")
    \end{lstlisting}
\end{minipage}
\hfill
\begin{minipage}{0.35\linewidth}
    \begin{lstlisting}[numbers=left,basicstyle=\tiny\ttfamily,numbers=none]
class exampleDataClass:                                                                                                      
    cls_var = "cls_data"                                                                        

    def __init__(self):                                                                         
        self.obj_var =
            "obj_data"   
                                                            
    def f(self):                                                                                
        return self.obj_var
    \end{lstlisting}
\end{minipage}
\end{frame}

\begin{frame}[fragile]{Инкапсуляция}
    \begin{lstlisting}
class Simple:
    """ Simple class with private attribute """
    def __init__(self, count, str):
        self.__private_attr = 20
        print self.__private_attr
 
s = Simple(1,'22')
print s.__private_attr
\end{lstlisting}
\end{frame}

\begin{frame}[fragile]{Наследование}
Синтаксис:
    \begin{lstlisting}[numbers=none]
class Derived(Base):
    pass
class Derived(module_name.Base):
    pass
class Derived(Base1, Base2, Base3):  
    pass
    \end{lstlisting}
Особенности:
\begin{itemize}
\item Все методы --- виртуальные;
\item Вызов метода базового класса: \lstinline$Base.method()$;
\item Порядок поиска атрибута:
  \begin{enumerate}
    \item
      Derived,
    \item Base1, затем рекурсивно в базовых классах,
    \item Base2, затем рекурсивно в базовых классах $\dots$
  \end{enumerate}
\end{itemize}
\end{frame}

\begin{frame}[fragile]{Наследование}
\begin{itemize}
\item type(object);
\item isinstance(object, classinfo);
\item issubclass(class, classinfo).
\end{itemize}
\end{frame}


\begin{frame}{Встроенные методы классов}
  \begin{itemize}
  \item \_\_name\_\_
  \item \_\_module\_\_
  \item \_\_dict\_\_
  \item \_\_bases\_\_
  \item \_\_doc\_\_
  \end{itemize}
\end{frame}

\begin{frame}{Встроенные методы объектов}
  \begin{itemize}
  \item \_\_dict\_\_
  \item \_\_class\_\_
  \item \_\_init\_\_
  \item \_\_del\_\_
  \item \_\_cmp\_\_
  \item \_\_hash\_\_
  \item \_\_getattr\_\_
  \item \_\_setattr\_\_
  \item \_\_delattr\_\_
  \item \_\_call\_\_
  \end{itemize}
\end{frame}

\begin{frame}[fragile]{Эмуляция последовательностей}
  \begin{minipage}{0.3\linewidth}
    \begin{itemize}
    \item \_\_len\_\_
    \item \_\_getitem\_\_
    \item \_\_setitem\_\_
    \item \_\_delitem\_\_
    \item \_\_getslice\_\_
    \item \_\_setslice\_\_
    \item \_\_delslice\_\_
    \item \_\_contains\_\_
    \end{itemize}
  \end{minipage}
  \hfill
  \begin{minipage}{0.65\linewidth}
    \begin{lstlisting}[basicstyle=\scriptsize\ttfamily]
import logging

class LoggingDict(dict):
    def __setitem__(self, key, value):
        logging.info(
            "Setting {k} to {v}".\
            format(k=key, v=value))
        return super(LoggingDict, self).\
        __setitem__(key, value)

logging.basicConfig(level=logging.INFO)

ld = LoggingDict()
ld["a"] = 123
      \end{lstlisting}
  \end{minipage}
\end{frame}

\begin{frame}{Приведение к базовым типам}
  \begin{itemize}
  \item \_\_repr\_\_
  \item \_\_str\_\_
  \item \_\_oct\_\_
  \item \_\_hex\_\_
  \item \_\_complex\_\_
  \item \_\_int\_\_
  \item \_\_long\_\_
  \item \_\_float\_\_
  \end{itemize}\
\end{frame}

\begin{frame}[fragile]{Пример перегрузки операторов}
    \begin{lstlisting}[basicstyle=\tiny\ttfamily]
class SignableMatrix:
    def __init__(self, array=[[]]):
        self.array = dc(array)
        dim_size = len(self.array)
        self.__row_signs = [False for i in range(dim_size)]
        self.__col_signs = [False for i in range(dim_size)]

    def __repr__(self):
        repr_str = "Matrix: [\n"
        for row in self.array:
            repr_str += "    {}\n".format(row)
        repr_str += "]\n"
        repr_str += "Row signs:\n    {}\n".format(self.__row_signs)
        repr_str += "Column signs:\n    {}".format(self.__col_signs)
        
        return repr_str

init_graph = [[1, 2, 3], [1, 2, 3], [1, 2, 3]]
init_matrix = SignableMatrix(init_graph)

print init_matrix
    \end{lstlisting}
\end{frame}

\begin{frame}{Остальные магические методы}
\begin{itemize}
  \item http://www.rafekettler.com/magicmethods.html
  \item \scriptsize{https://docs.python.org/2/reference/datamodel.html\#special-method-names}
\end{itemize}
\end{frame}

\begin{frame}{Декораторы классов}
  \begin{itemize}
  \item @staticmethod
  \item @classmethod
  \item @property
  \item @setter
  \item @deleter
  \end{itemize}
\end{frame}

\begin{frame}{Исключения}
Обработка исключительных ситуаций --- \\
механизм языков программирования, \\
предназначенный для описания реакции программы на возможные проблемы (исключения), \\
которые могут возникнуть при выполнении программы \\
и приводят к невозможности (бессмысленности) \\
дальнейшей отработки программой её базового алгоритма.~\footnote[frame]{
http://en.wikipedia.org/wiki/Exception\_handling}
\end{frame}

\begin{frame}[fragile]{Как ловить исключение}
  \begin{lstlisting}
for arg in sys.argv[1:]:
  try:
      f = open(arg, 'r')
  except IOError:
      print 'cannot open', arg
  else:
      print arg, 'has', len(f.readlines()), 'lines'
      f.close()
  \end{lstlisting}
\end{frame}

\begin{frame}[fragile]{Как сгенерировать исключение}
  \begin{lstlisting}
try:
    raise Exception('spam', 'eggs')
except Exception as inst:
    print type(inst)    
    print inst.args     
    print inst          
    x, y = inst.args
    print 'x =', x
    print 'y =', y
  \end{lstlisting}
\end{frame}

\begin{frame}[fragile]{Безопасное деление}
  \begin{lstlisting}
def divide(x, y):
     try:
         result = x / y
     except ZeroDivisionError:
         print "division by zero!"
     else:
         print "result is", result
     finally:
         print "executing finally clause"
  \end{lstlisting}
\end{frame}


\begin{frame}[fragile]{Стандартные исключения}
\begin{minipage}{0.45\linewidth}
  \begin{lstlisting}[basicstyle=\tiny\ttfamily]
BaseException
 +-- SystemExit
 +-- KeyboardInterrupt
 +-- GeneratorExit
 +-- Exception
      +-- StopIteration
      +-- StandardError
      +-- Warning


 +-- Warning
      +-- DeprecationWarning
      +-- PendingDeprecationWarning
      +-- RuntimeWarning
      +-- SyntaxWarning
      +-- UserWarning
      +-- FutureWarning
      +-- ImportWarning
      +-- UnicodeWarning
      +-- BytesWarning
\end{lstlisting}
\end{minipage}
\hfill
\begin{minipage}{0.5\linewidth}
  \begin{lstlisting}[basicstyle=\tiny\ttfamily]
+-- StandartError
    +-- BufferError
    +-- ArithmeticError
    |    +-- FloatingPointError
    |    +-- OverflowError
    |    +-- ZeroDivisionError
    +-- AssertionError
    +-- AttributeError
    +-- EnvironmentError
    |    +-- IOError
    |    +-- OSError ...
    +-- EOFError
    +-- ImportError
    +-- LookupError
    |    +-- IndexError
    |    +-- KeyError
    +-- MemoryError
    +-- NameError
    |    +-- UnboundLocalError
    +-- ReferenceError
    +-- RuntimeError
    |    +-- NotImplementedError
    +-- SyntaxError
    |    +-- IndentationError
    |         +-- TabError
    +-- SystemError
    +-- TypeError
    +-- ValueError
         +-- UnicodeError ...
\end{lstlisting}
\end{minipage}
\end{frame}

\begin{frame}{Полезные ссылки}
  \begin{itemize}
    \item https://docs.python.org/2/tutorial/classes.html
    \item https://docs.python.org/2/library/exceptions.html
    \item http://www.rafekettler.com/magicmethods.html
      \scriptsize{
      \item https://docs.python.org/2/reference/datamodel.html\#special-method-names
      \item http://www.ibm.com/developerworks/ru/library/l-python\_part\_6/
      \item https://www.ibm.com/developerworks/ru/library/l-python\_part\_7/ 
      }
    % \item http://zetcode.com/lang/python/functions/
    % \item https://docs.python.org/2/library/functions.html
    % \item https://docs.python.org/2/howto/functional.html
    % \item https://docs.python.org/2/tutorial/modules.html
    % \item https://docs.python.org/2.7/library/
  \end{itemize}
\end{frame}

\begin{frame}{The End}
  \begin{itemize}
  \item Вопросы и предложения;
  \item Планы на май.
  \end{itemize}
\end{frame}


\end{document}