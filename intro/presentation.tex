\documentclass[hyperref={pdftex,unicode}]{beamer}

\usepackage[T2A]{fontenc}
\usepackage[utf8]{inputenc}
\usepackage[russian]{babel}
\usepackage{cmap}

\usepackage{xcolor}

% \usepackage{helvet}
\usepackage{pscyr}

\usepackage{multicol}

% \usepackage{amssymb,amsfonts,amsmath,mathtext}
% \usepackage{cite,enumerate,float}

\usepackage{listings}

\graphicspath{{images/}}
\usetheme{boxes}
\usecolortheme{whale}
\useinnertheme{default}
\useoutertheme{default}
\usefonttheme{professionalfonts}

\definecolor{Blue}{RGB}{55,110,160}
\definecolor{Yellow}{RGB}{255,210,65}
\definecolor{Green}{RGB}{75,120,60}
\definecolor{LightGray}{RGB}{205,205,205}

\setbeamercolor*{palette primary}{fg=white,bg=Blue}

\setbeamercolor*{enumerate item}{fg=Yellow}
\setbeamercolor*{enumerate subitem}{fg=Yellow}
\setbeamercolor*{enumerate subsubitem}{fg=Yellow}

\setbeamercolor*{itemize item}{fg=Yellow}
\setbeamercolor*{itemize subitem}{fg=Yellow}
\setbeamercolor*{itemize subsubitem}{fg=Yellow}

\lstset{
  language=python,
  basicstyle=\small\ttfamily,
  keywordstyle=\bfseries\color{Blue},
  commentstyle=\color{Green},
  numberstyle=\footnotesize\color{LightGray},
  numbersep=0.8em,
  stepnumber=1,
  keepspaces=true,
  breaklines=true,
  aboveskip=0.5\baselineskip,
  belowskip=0.5\baselineskip}


\title{Python: functions}
\author{meequz@gmail.com \\ budnyjj@pirates.by}
\date{}

\begin{document}

\begin{frame}
  \maketitle
\end{frame}

\begin{frame}{Обзор курса}
  \begin{itemize}
    \item Базовые типы
    \item Контейнеры
    \item Функции
    \item Классы
    \item Исключения
    \item IO
  \end{itemize}
\end{frame}

\begin{frame}{Ожидания и реальность}
  $$
  \mbox{разочарование} = \dfrac{\mbox{ожидания}}{\mbox{реальность}} 
  $$
\end{frame}

\begin{frame}{Сегодня}
  \begin{itemize}
  \item Обзор языка
  \item Hello, world!
  \item Базовые типы данных
  \item Условный оператор
  \item Операторы цикла
  \end{itemize}
\end{frame}

\begin{frame}{Особенности языка Python}
  \begin{itemize}
    \item Интерпретируемый
    \item Высокоуровневый
    \item Динамически типизируемый
    \item Поддержкивает ООП
  \end{itemize}
\end{frame}

\begin{frame}{Версии Python}
  \centering
  ``Python 2.x is legacy, \\
  Python 3.x is the present
  and future of the language'' \footnote[frame]{
    https://wiki.python.org/moin/Python2orPython3}
\end{frame}

\begin{frame}{Getting help}
  \begin{itemize}
    \item https://docs.python.org
    \item opennet.ru/docs/RUS/python/
    \item help()
  \end{itemize}
\end{frame}

\begin{frame}[fragile]{Hello, world!}
  \begin{itemize}
  \item Interactive
    \begin{lstlisting}
python
>>> print "Hello, world!"
    \end{lstlisting}
  \item Script hello.py:
    \begin{lstlisting}[numbers=left]
#!/usr/bin/python2
print "Hello, world!"
    \end{lstlisting}

  \begin{minipage}{0.4\linewidth}
     \begin{lstlisting}[language=bash]
chmod +x hello.py
./hello.py
     \end{lstlisting}
   \end{minipage}
   \hfill or \hfill
   \begin{minipage}{0.4\linewidth}
     \begin{lstlisting}[language=bash]
python2 hello.py
     \end{lstlisting}
   \end{minipage}

  \end{itemize}
\end{frame}

\begin{frame}[fragile]{import this}
  \begin{center}
    \texttt{python -c "import this"}
  \end{center}
\end{frame}

\begin{frame}{Zen of Python}
\footnotesize{
The Zen of Python, by Tim Peters.

Beautiful is better than ugly.

Explicit is better than implicit.

Simple is better than complex.

Complex is better than complicated.

Flat is better than nested.

Sparse is better than dense.

Readability counts.

Special cases aren't special enough to break the rules.

Although practicality beats purity.

Errors should never pass silently.

Unless explicitly silenced.

In the face of ambiguity, refuse the temptation to guess.

There should be one --- and preferably only one --- obvious way to do it.

Although that way may not be obvious at first unless you're Dutch.

Now is better than never.

Although never is often better than *right* now.

If the implementation is hard to explain, it's a bad idea.

If the implementation is easy to explain, it may be a good idea.

Namespaces are one honking great idea --- let's do more of those!}
\end{frame}

\begin{frame}{Идентификаторы}
  \begin{itemize}
    \item A-Z, a-z, 0-9, \_
    \item Case sensitive
  \end{itemize}
\end{frame}

\begin{frame}{Стандартные типы данных}
  \begin{itemize}
    \item Boolean
    \item Numeric: int, float, long, complex
    \item String
    \item List
    \item Tuple
    \item (x)Range
    \item Dictionary\footnote[frame]{
        Существуют различия в типах данных в разных версиях Python: \\
        https://docs.python.org/2.7/library/stdtypes.html \\
        https://docs.python.org/3.4/library/stdtypes.html}
  \end{itemize}
\end{frame}

\begin{frame}[fragile]{Базовые типы данных}
  \begin{lstlisting}[numbers=left]
#!/usr/bin/python

logic   = True         # A boolean assignment
counter = 103          # An integer 
miles   = 1000.0       # A floating point
cmplx   = 1 + 1j       # A complex 
name    = "John"       # A string

print logic, not Logic
print counter, counter * miles
print miles, miles / counter, miles // counter
print cmplx, cmplx.conjugate()
print name, "|".join(name)
\end{lstlisting}
\end{frame}

\begin{frame}[fragile]{Преобразование типов}
  \begin{lstlisting}[numbers=left]
type(x)
int(x[, base])
long(x[, base])
float(x)
complex(real[,imag])
str(x)
repr(x)
eval(x)
chr(x)
unichr(x)
ord(x)
\end{lstlisting}
\end{frame}

\begin{frame}[fragile]{Операторы}
\begin{lstlisting}
+, -, *, /, %, **, //    # arithmetical
==, !=, <, >, >=, <=     # comparison
and, or, not             # logical
in                       # membership
is                       # identity
\end{lstlisting}
\end{frame}

\begin{frame}[fragile]{Пара слов про контейнеры}
  \begin{itemize}
    \item List
      \begin{lstlisting}[numbers=left]
l = [1, 2]
l.append(3)
print l
l.append(l)
print l
      \end{lstlisting}
    \item Dict
      \begin{lstlisting}[numbers=left]
d = {"one": 1, "two": 2,}
d["three"] = 3
print d
del(d["two"]
print d
      \end{lstlisting}
  \end{itemize}
\end{frame}

\begin{frame}[fragile]{Примечания}
  \begin{itemize}
    \item Boolean, Numeric, Complex, String, Tuple are \textbf{immutable}
    \item List, Dictionary are \textbf{mutable}
    \item \lstinline$'foo' == "foo"$
    \item Docstring:
      \begin{lstlisting}
"""
This is a docstring example.

It is useful mainly for documentation purposes.
"""
      \end{lstlisting}
  \end{itemize}
\end{frame}

\begin{frame}[fragile]{Условный оператор}
  \begin{lstlisting}[numbers=left]
x = int(raw_input("Please enter an integer: "))
if x < 0:
    x = 0
    print "Negative changed to zero"
elif x == 0:
    print "Zero"
elif x == 1:
    print "Single"
else:
    print "More"
  \end{lstlisting}
\end{frame}

\begin{frame}[fragile]{Оператор цикла for}
  \begin{lstlisting}[numbers=left]
words = ["cat", "window", "defenestrate"]
for w in words:
    print w, len(w)

# range(5)         == [0, 1, 2, 3, 4]
# range(3,10)      == [3, 4, 5, 6, 7, 8, 9]
# range(3, 10, 2)  == [3, 5, 7, 9]
# range(3, 10, -2) == [9, 7, 5, 3]

for i in range(len(words)):       # ugly
    print i, words[i]

for i, word in enumerate(words):  # much better
    print i, word

\end{lstlisting}
\end{frame}

\begin{frame}[fragile]{Оператор цикла for}
  \begin{lstlisting}[numbers=left]
for n in range(2, 10):
     for x in range(2, n):
         if n % x == 0:
             print n, "equals", x, "*", n/x
             break
     else:
         print n, "is a prime number"
  \end{lstlisting}
\end{frame}

\begin{frame}[fragile]{Оператор цикла for}
  \begin{lstlisting}[numbers=left]
for num in range(2, 10):
    if num % 2 == 0:
         print "Found an even number", num
         continue
    print "Found a number", num
  \end{lstlisting}
\end{frame}

\begin{frame}[fragile]{Оператор цикла while}
  \begin{lstlisting}[numbers=left]
while True:
    pass
  \end{lstlisting}
\end{frame}

\begin{frame}[fragile]{import math}
\begin{lstlisting}[numbers=left]
import math
dir(math)
help(math)
\end{lstlisting}
\end{frame}

\begin{frame}{Задачи на дом}
\begin{enumerate}
  \item n-тоe число Фибоначчи
  \item Поиск наименьшего общего кратного двух чисел
  \item Решение квадратного уравнения
  \item Гипотеза Коллатца\footnote[frame]{http://euler.jakumo.org/problems/view/14.html}
\end{enumerate}
\end{frame}

\begin{frame}[fragile]{Пример решения задачи <<n-тоe число Фибоначчи>>\footnote[frame]{
http://stackoverflow.com/questions/494594/how-to-write-the-fibonacci-sequence-in-python}}
\begin{equation*}
  F_n = \left\{
    \begin{array}{l l}
      0 & \quad \text{при n = 0;}\\
      1 & \quad \text{при n = 1;}\\
      F_{n-1} + F_{n-2} & \quad \text{при n > 1.}
    \end{array} \right.
\end{equation*}
\smallskip

\begin{lstlisting}[basicstyle=\footnotesize\ttfamily,numbers=left]
def fib(n):
""" Return n-th number in Fibonacci sequence. """
    if n == 0: return 0
    elif n == 1: return 1
    else: return fib(n-1) + fib(n-2)

if __name__ == "__main__":
    reqNumber = int(raw_input(
    "Enter index of requested Fibonacci number: "))
    print("Your number is:", fib(reqNumber))
\end{lstlisting}
\end{frame}

\begin{frame}{Задача <<n-тоe число Фибоначчи>>}
  \begin{itemize}
  \item Найти и исправить ошибки в коде на предыдущем слайде
  \item Решить задачу более оптимальным методом
  \end{itemize}
\end{frame}

\begin{frame}{Продолжение следует\dots}
  Через неделю --- контейнеры:
  \begin{itemize}
  \item Строки
  \item Списки
  \item Генераторы
  \item Словари
  \end{itemize}
\end{frame}


\end{document}